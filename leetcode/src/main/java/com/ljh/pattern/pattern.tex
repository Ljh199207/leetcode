 设置模式总结:
 一 :设计原则
 1,单一职责原则,一个类,只有一个引起它变化的原因,应该只有一个职责,每一个职责都是变化的轴线,如果一个类有一个以上的职责,
 这些职责就耦合在一起,这会导致脆弱的设计,当一个职责发生变化,可能引起其它的职责。
 2,开闭原则,对扩展开放,对修改关闭。对程序需要进行拓展的时候,不能修改原有的设计,实现一个热插拔的效果,就是为了使程序扩展性好,
 易于维护和升级。
 3,里氏替换原则,任何基类出现的地方,子类一定可以出现,里氏对开闭的补充。
 4,依赖倒转原则,就是依赖于抽象,不要依赖于实现,
 5,接口隔离原则,使用多个隔离的接口,比使用单个接口要好,还是一个降低类之间耦合度的意思。
 6,合成服用原则,一个新的对象里通过关联关系来使用已有的对象,使之成为新对象的一部分,多使用组合、聚合关系,少用继承。
 7,迪米特法则,最少知道法则,一个实体应当尽量少的与其他实体之间发生相互作用,使得系统功能模块相对独立
